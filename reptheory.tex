\documentclass[a4paper]{article}

\title{Representations}
\author{Notes taken by Harry Partridge}
\date{Term 1 2021}
\def\descrip{Based on course notes by Ed Segal at the Imperial College London}
 
\RequirePackage{etex}

\usepackage{alltt}
\usepackage{amsfonts}
\usepackage{amsmath}
\usepackage{amssymb}
\usepackage{amsthm}
\usepackage{booktabs}
\usepackage{caption}
\usepackage{enumitem}
\usepackage{fancyhdr}
\usepackage{graphicx}
\usepackage[hidelinks]{hyperref}
\usepackage{lmodern}
\usepackage{mathdots}
\usepackage{mathtools}
\usepackage{microtype}
\usepackage{multirow}
\usepackage{pdflscape}
\usepackage{siunitx}
\usepackage{slashed}
\usepackage{tabularx}
\usepackage{tikz}
\usepackage{tkz-euclide}
\usepackage[normalem]{ulem}
\usepackage[all]{xy}
\usepackage{imakeidx}

\makeatletter   
\renewcommand\maketitle{
\begin{titlepage}
    \begin{center}
    \vspace*{15em}
    {\fontsize{28}{38}\selectfont \textit{\@title}}
    \vspace{2em}

    \begin{Large}
        {\descrip}
    \end{Large}

    \vspace{1em}
    \textbf{\@author}
    \vfill
    \@date
    \end{center}
\end{titlepage}}
\makeatother

\theoremstyle{definition}
\newtheorem*{axiom}{Axiom}
\newtheorem*{claim}{Claim}
\newtheorem*{cor}{Corollary}
\newtheorem*{defi}{Definition}
\newtheorem*{eg}{Example}
\newtheorem*{lemma}{Lemma}
\newtheorem*{prop}{Proposition}
\newtheorem*{thm}{Theorem}
\newtheorem*{princ}{Principle}

\pagestyle{fancyplain}
\lhead{\textit{\nouppercase{\leftmark}}}
\rhead{MATH3001}
\cfoot{\thepage}

\usepackage{geometry}
\geometry{a4paper, portrait, margin=4.5cm}

\DeclareMathOperator{\im}{im}
\DeclareMathOperator{\id}{id}
\DeclareMathOperator{\tr}{tr}
\DeclareMathOperator{\ch}{char}
\DeclareMathOperator{\inv}{inv}
\DeclareMathOperator{\Aut}{Aut} 
\DeclareMathOperator{\End}{End}
\DeclareMathOperator{\Ad}{Ad}
\DeclareMathOperator{\ad}{ad}
\DeclareMathOperator{\Der}{Der}


\newcommand{\abs}[1]{\left\lvert #1 \right\rvert}
\newcommand{\ket}[1]{\left\lvert #1 \right\rangle}
\newcommand{\bra}[1]{\left\langle #1 \right\rvert}
\newcommand{\anglet}[2]{\left\langle #1 , #2 \right\rangle}
\newcommand{\bracket}[1]{\left[ #1 \right]}
\newcommand{\norm}[1]{\left\lVert #1\right\rVert}
\newcommand{\set}[1]{\left\{ #1 \right\}}
% \renewcommand{\vec}[1]{\boldsymbol{\mathbf{#1}}} 

\renewcommand{\H}{\mathcal{H}}
\newcommand{\C}{\mathbb{C}}
\newcommand{\class}{\mathcal{C}}
\newcommand{\N}{\mathbb{N}}
\newcommand{\Q}{\mathbb{Q}}
\newcommand{\R}{\mathbb{R}}
\newcommand{\Z}{\mathbb{Z}}
\newcommand{\Zplus}{\mathbb{Z}^+}
\newcommand{\F}{\mathbb{F}} 

\renewcommand{\a}{\mathfrak{a}}
\newcommand{\g}{\mathfrak{g}}
\newcommand{\gl}{\mathfrak{gl}}
\renewcommand{\sl}{\mathfrak{sl}}
\newcommand{\su}{\mathfrak{su}}
\newcommand{\so}{\mathfrak{so}}
\newcommand{\h}{\mathfrak{h}}



\begin{document}

\maketitle 
\tableofcontents

\newpage
\section{Representations}
\subsection{Definitions}
\begin{itemize}
    \item For any group G, an $n$ dimensional \textbf{matrix representation} of a group G is a homomorphism $$\rho: G \to GL_n(\C).$$
    \item A representation is \textbf{faithful} if it is injective.
    \item Define $c_P: GL_n(\C) \to GL_n(\C)$ by $c_P(M) = P^{-1}MP$. Two $n$ dimensional representations $\rho_1$ and $\rho_2$ are said to be \textbf{equivalent} if $\rho_2 = c_P \circ \rho_1$ for some $P$.
    \item An $n$ dimensional \textbf{representation} of a group G is $(\rho, V)$ where $\rho: G \to GL(V)$ is a homomorphism and $V$ is an $n$ dimensional complex vector space.
    \item For any $n$ dimensional complex vector space $V$ with a basis $\mathcal{B}$, there is an isomorphism $\phi^\mathcal{B}: GL(V) \to GL_n(\C)$. Therefore, for every representation $\rho: G \to GL(V)$, an \textbf{associated matrix representation} is $\rho^\mathcal{B} = \phi^\mathcal{B} \circ \rho: G \to GL(V) \to GL_n(\C)$. 
    \item Two matrix representations are equivalent iff they are both associated to the same representation.
\end{itemize}

\subsection{Basic representations}
\begin{itemize}
    \item For any group G, the trivial representation of dimension $n$ is $\rho: g \mapsto I_V$, where $I_V$ is the identity map on $V$.
    \item The \textbf{permutation representation} of $H \subset S_n$ is constructed by considering the action of $H$ on a set of $n$ basis vectors for $V$, and extending this to a linear map in $GL(V)$.
    \item The \textbf{regular representation} of $G$ is found by considering the permutation action of $G$ on itself.
    \item If $f: H \to G$ is a homomorphism, and $\rho: G \to GL(V)$, then $\rho \circ f$ is a representation of $H$ on $V$. (the composition of homomorphisms is a homomorphism)
    \item The sign representation of a group is $\rho : g \mapsto \text{sgn}(\sigma_g)$ where $\sigma_g$ is the permutation corresponding to $g$.
\end{itemize}

\subsection{G-Linear maps}
\begin{itemize}
    \item For two representations $(\rho_1, V)$ and $(\rho_2, W)$ of $G$, a G-linear map is a linear map $f: V \to W$ such that for all $g \in G$, $$f \circ \rho_1(g) = \rho_2(g) \circ f.$$
    \item Two representations of G are said to be isomorphic if there is a G-linear map between them which is also an isomorphism.
    \item Isomorphism of representations is the same thing as equivalence of matrix representations in the sense that if two representations are isomorphic, then their associated matrix representations will be equivalent.
\end{itemize}

\subsection{Subrepresentations}
\begin{itemize}
    \item A subrepresentation of $(\rho, V)$ is a representation $(\rho_{|W}, W)$, where $W$ is a subspace of $V$.
    \item If $f: V \to W$ is a G-linear map between two representations, then Ker$(f)$ is a subrepresentation of $V$ and Im$(f)$ is a subrepresentation of $W$. 
    \item Applying the above, we see that for a given $\rho$, if $W$ is an eigenspace of $V$ for every $\rho(g)$, then $W$ is a subrepresentation of $V$.
\end{itemize}

\subsection{Maschke's Theorem}
\begin{itemize}
    \item For any two representations $(\rho_V, V)$ and $(\rho_W, W)$, there is a natural representation on $V \oplus W$: $\rho_{V \oplus W}(g): (x, y) \mapsto (\rho_V(g)(x), \rho_W(g)(y))$.
    \item If two bases $\{a_1, \hdots, a_n\}$ and $\{b_1, \hdots, b_n\}$ are chosen for $V$ and $W$, then a natural basis for $V \oplus W$ is $\{(a_1, 0), \hdots, (a_n,0), (0, b_1), \hdots, (0,b_n)\}$. Then if $\rho_V(g)$ and $\rho_W(g)$ have associated matrices $M$ and $N$, then $$\rho_{V \oplus W}(g) = \begin{pmatrix}M & 0 \\ 0 & N\end{pmatrix}.$$
    \item If $(\rho, V)$ is a representation with subrepresentations $W$ and $U$, where $W \cup W = \{0\}$ and $\dim U + \dim W = \dim V$, then $\rho_{V \oplus W}$ is isomorphic to $\rho$.
    \item If $W \subset V$ is a subspace of $V$, and $f : V \to W: x \mapsto x \forall x \in W$ is a linear map, then 
    \item Maschke's theorem
\end{itemize}

\newpage
\section{questions}
\begin{itemize}
    \item Why do we sometimes treat a vector space as a representation? (i.e. there are statements like `$V$ is a 3 dimensional representation of $G$') Can we not have different homomorphisms from the same group to the vector space? i.e. $C_2$ to $GL(C^2)$ could map $\sigma$ to $$\begin{pmatrix}0 & 1 \\ 1 & 0\end{pmatrix} \text{ or } \begin{pmatrix}-1 & 0 \\ 0 & -1\end{pmatrix}.$$ Are these not different? 
\end{itemize}

\end{document}