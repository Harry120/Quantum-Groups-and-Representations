\documentclass[a4paper]{article}

\title{MATH3001}
\author{Harry Partridge}
\date{Term 1 2021}
\def\descrip{Quantum Mechanics and Representation Theory}

\RequirePackage{etex}

\usepackage{alltt}
\usepackage{amsfonts}
\usepackage{amsmath}
\usepackage{amssymb}
\usepackage{amsthm}
\usepackage{booktabs}
\usepackage{caption}
\usepackage{enumitem}
\usepackage{fancyhdr}
\usepackage{graphicx}
\usepackage[hidelinks]{hyperref}
\usepackage{lmodern}
\usepackage{mathdots}
\usepackage{mathtools}
\usepackage{microtype}
\usepackage{multirow}
\usepackage{pdflscape}
\usepackage{siunitx}
\usepackage{slashed}
\usepackage{tabularx}
\usepackage{tikz}
\usepackage{tkz-euclide}
\usepackage[normalem]{ulem}
\usepackage[all]{xy}
\usepackage{imakeidx}

\makeatletter   
\renewcommand\maketitle{
\begin{titlepage}
    \begin{center}
    \vspace*{15em}
    {\fontsize{28}{38}\selectfont \textit{\@title}}
    \vspace{2em}

    \begin{Large}
        {\descrip}
    \end{Large}

    \vspace{1em}
    \textbf{\@author}
    \vfill
    \@date
    \end{center}
\end{titlepage}}
\makeatother

\theoremstyle{definition}
\newtheorem*{axiom}{Axiom}
\newtheorem*{claim}{Claim}
\newtheorem*{cor}{Corollary}
\newtheorem*{defi}{Definition}
\newtheorem*{eg}{Example}
\newtheorem*{lemma}{Lemma}
\newtheorem*{prop}{Proposition}
\newtheorem*{thm}{Theorem}
\newtheorem*{princ}{Principle}

\pagestyle{fancyplain}
\lhead{\textit{\nouppercase{\leftmark}}}
\rhead{MATH3001}
\cfoot{\thepage}

\usepackage{geometry}
\geometry{a4paper, portrait, margin=4.5cm}

\DeclareMathOperator{\im}{im}
\DeclareMathOperator{\id}{id}
\DeclareMathOperator{\tr}{tr}
\DeclareMathOperator{\ch}{char}
\DeclareMathOperator{\inv}{inv}
\DeclareMathOperator{\Aut}{Aut} 
\DeclareMathOperator{\End}{End}
\DeclareMathOperator{\Ad}{Ad}
\DeclareMathOperator{\ad}{ad}
\DeclareMathOperator{\Der}{Der}


\newcommand{\abs}[1]{\left\lvert #1 \right\rvert}
\newcommand{\ket}[1]{\left\lvert #1 \right\rangle}
\newcommand{\bra}[1]{\left\langle #1 \right\rvert}
\newcommand{\anglet}[2]{\left\langle #1 , #2 \right\rangle}
\newcommand{\bracket}[1]{\left[ #1 \right]}
\newcommand{\norm}[1]{\left\lVert #1\right\rVert}
\newcommand{\set}[1]{\left\{ #1 \right\}}
% \renewcommand{\vec}[1]{\boldsymbol{\mathbf{#1}}} 

\renewcommand{\H}{\mathcal{H}}
\newcommand{\C}{\mathbb{C}}
\newcommand{\class}{\mathcal{C}}
\newcommand{\N}{\mathbb{N}}
\newcommand{\Q}{\mathbb{Q}}
\newcommand{\R}{\mathbb{R}}
\newcommand{\Z}{\mathbb{Z}}
\newcommand{\Zplus}{\mathbb{Z}^+}
\newcommand{\F}{\mathbb{F}} 

\renewcommand{\a}{\mathfrak{a}}
\newcommand{\g}{\mathfrak{g}}
\newcommand{\gl}{\mathfrak{gl}}
\renewcommand{\sl}{\mathfrak{sl}}
\newcommand{\su}{\mathfrak{su}}
\newcommand{\so}{\mathfrak{so}}
\newcommand{\h}{\mathfrak{h}}



\begin{document}

\maketitle

\tableofcontents

\newpage
\section{Quantum Physics}
\subsection{Axioms of Quantum Mechanics}
\subsubsection{States and Observables}
Classically, a particle is characterised by its properties - a particle is an entity with a specified position, momentum, energy etc. One of the key features of Quantum Mechanics is that it allows for discussion of a particle without reference to these properties. This is facilitated by introducing the notion of a state vector, which resides in a physical$^\dagger$ Hilbert space. 

\begin{axiom}[Representation of State] 
    Any physical system has a corresponding Hilbert Space $\h$. The state of a particle at time $t$ is represented by a unit vector $\ket{\psi(t)} \in \h$.
\end{axiom}

Instead of characterising the particle, quantum mechanics postulates that observable properties are hermitian operators with an orthogonally normalisable basis of eigenstates (an eigenbasis). The notion that a particle has a specific value for a given observable is then discarded - instead, we expand $\ket{\psi}$ in the eigenbasis corresponding to $\Omega$, and say that the probability of measuring the particle in the eigenstate $\ket{\omega}$ is the inner product $\anglet{\omega}{\psi}$ of $\ket{\psi}$ with $\ket{\omega}$. 

\begin{axiom}[Observable]
    Every observable property of the state is described by a hermitian operator $\Omega: \h \to \h$, with an orthogonally normalisable basis of eigenvectors $\ket{\omega}$. 
\end{axiom}


\begin{axiom}[Measurement]
    When a particle is in the state $\ket{\psi} = \sum \alpha_\omega \ket{\omega}$ with $\anglet{\psi}{\psi} = 1$, measurement of $\Omega$ will yield the value $\omega$ with probability $\abs{\anglet{\omega}{\psi}}^2 = |\alpha_\omega|^2$. This measurement will cause the state of the particle to change from $\ket{\psi}$ to $\ket{\omega}$. 
\end{axiom}

This ensures that repeated measurements are consistent - if we measure the value of an observable to be $\omega$, the state of the particle will then be the eigenvector $\ket{\omega}$. If we then perform another measurement, the probability that the value of the observable is still $\omega$ is $\anglet{\omega}{\omega} = 1$.

We now list a few common operators.

\begin{defi}[Position operator] 
    We define the position operator $X$ by requiring that the position vector $\ket{x}$ (corresponding to the particle being at position $x$) is an eigenvector of $X$. i.e.  $X\ket{x} = x\ket{x}$. The wave function $\psi(x)$ is defined by the inner product of the state with $\ket{x}$ eigenbasis: $\psi(x) = \anglet{x}{\psi}$.
\end{defi}

\begin{defi}[Momentum operator] 
    The momentum operator $P$ can be defined by its action on wave functions. We have $\bra{x}P\ket{\psi} = -i\hbar \psi'(x)$.
\end{defi}

\begin{defi}[Hamiltonian operator] 
    The hamiltonian operator $H$ is defined by its action on wave functions: $\bra{x}P\ket{\psi} = -i\hbar \psi'(x)$.
\end{defi}
 
\subsubsection{Schrodinger Equation}
The final axiom of quantum mechanics describes how a state vector evolves through time.

\begin{axiom}[Schrodinger Equation]
    The state vector $\ket{\psi(t)}$ obeys $$i\hbar \frac{d}{dt}\ket{\psi(t)} = H\ket{\psi(t)},$$
    where $H$ is the energy operator (the Hamiltonian) of the system.
\end{axiom}

\begin{eg}[Hamiltonian for a free particle]
    For a particle that is not under the influence of 
\end{eg}

\subsection{Interpretations}
\subsubsection{Copenhagen}
\subsubsection{Many Worlds}
\subsection{Expectation and Uncertainty}
\subsubsection{Expectation, Uncertainty}
\subsubsection{Ehrenfest's Theorem}
\subsubsection{Heisenberg's Uncertainty Principle}
\subsubsection{Probability Current}
\subsection{Problems in One dimension}
\subsubsection{Free Particle}
\subsubsection{Gaussian Wavepacket}
\subsubsection{Particle in a Box}
\subsubsection{Finite Square Well}
\subsubsection{Step Potential}
\subsubsection{Harmonic Oscillator}
\subsubsection{Scattering}
\subsection{System with N degrees of freedom}
\subsubsection{N particles in one dimension}
\subsubsection{N particles in M dimensions}
\subsection{Hydrogen Atom}
\newpage
\section{Lie Theory}
\section{Category Theory}



what is the wave function collapse?
 - properties are always found to be discrete, not smeared out




\end{document}